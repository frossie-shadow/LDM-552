\documentclass[DM,lsstdraft,STS,toc]{lsstdoc}
\usepackage{enumitem}
\input meta.tex

\begin{document}

\def\product{Qserv}

\setDocCompact{true}

\title[STS for \product]{\product{} Software Test Specification}

\author{Fritz Mueller}
\setDocRef{\lsstDocType-\lsstDocNum}
\date{\vcsdate}

\setDocAbstract{
This document describes the detailed test specification for the \product{}.
}

\setDocChangeRecord{%
    \addtohist{1}{\vcsdate}{Initial release.}{FM}
}

\maketitle

\section{Introduction}
\label{sec:intro}

This document specifies the test procedure for \product{}. \product{} is a distributed
shared-nothing RDBMS which will host LSST catalogs.

\subsection{Objectives}
\label{sec:objectives}

This document builds on the description of LSST Data Management's approach to
testing as described in \citeds{LDM-503} to describe the detailed tests that
will be performed on the \product{} as part of the verification of the DM system.

It identifies test designs, test cases and procedures for the tests, and the
pass/fail criteria for each test.

\subsection{Scope}
\label{sec:scope}

This document describes the test procedures for the following components of
the LSST system (as described in \citeds{LDM-148}):

\begin{itemize}
  \item{Parallel Distributed Database (Qserv)}
\end{itemize}

\subsection{Applicable Documents}
\label{sec:docs}

\addtocounter{table}{-1}

\begin{tabular}[htb]{l l}
  \citeds{LDM-135} & LSST Qserv Database Design \\
  \citeds{LDM-294} & LSST DM Organization \& Management \\
  \citeds{LDM-502} & The Measurement and Verification of DM Key Performance Metrics \\
  \citeds{LDM-503} & LSST DM Test Plan \\
  \citeds{LDM-555} & LSST DM Database Requirements \\
\end{tabular}

\subsection{References}
\label{sec:references
}
\renewcommand{\refname}{}
\bibliography{lsst,refs,books,refs_ads}

\newpage
\section{Approach}
\label{sec:approach}

The approaches taken for the tests described here are:

\begin{itemize}

  \item{Ongoing inspection of design documents, code, and CI system logs to verify that \product{} design
  and implementation meet DM software quality standards in general, and requirements as expressed in
  \citeds{LDM-555} in particular;}

  \item{Ongoing deployment and continuous operation of \product{} in a Prototype Data Access Center
  (PDAC) in order to assess basic reliability, fitness for purpose, and integration with adjacent
  subsystems;}

  \item{Annual deployment of \product{} to test clusters, followed by synthesis and ingestion
  of test datasets and scripted performance/load/stress testing. The cluster size/capabilities and the
  scale of the synthetic test dataset are both evolved along a path toward anticipated LSST operational
  scale.}

\end{itemize}

\subsection{Tasks and criteria}
\label{sec:tasks}

\product{} is a containerized, distributed, Linux application, which is deployed on machine clusters.
At the scales to be tested, these clusters are comprised of one to several head ("czar") nodes and
additionally on the order of tens to hundreds of shard ("worker") nodes, interconnected locally via a
high-performance network. Head and shard nodes are provisioned each with on the order of 10s of gigabytes of
RAM, and each with on the order of 10s of terabytes of locally attached storage.

Ongoing deployment, continuous operation, and integration tests are carried out on machines within the
Prototype Data Access Center (PDAC), a dedicated machine cluster physically located at NCSA's National
Peta-scale Compute Facility, maintained by NCSA staff. Catalog datasets which are maintained within
the PDAC Qserv instance and which are used for this testing include, simultaneously:

\begin{itemize}
  \item{An LSST stack reprocessed version of the SDSS Stripe 82 catalog ({\textasciitilde{}}10 TB);}
  \item{IRSA AllWISE and NEOWISE catalogs ({\textasciitilde{}}50 TB);}
  \item{An LSST stack reprocessed version of the HSC catalog (scheduled; {\textasciitilde{}}50 TB).}
\end{itemize}

Tasks required for these tests include periodic update of the software deployed on the PDAC,
periodic ingest of additional test datasets, and inter-operation with adjacent subsystems.  Uptime
is monitored cumulatively throughout these activities to gain quantitative insight into system
stability and reliability.

Scaling, load, and stress testing are carried out on an additional machine cluster located
CC-IN2P3 in Lyon, maintained by CC-IN2P3 staff. Scaling tests are run annually, by issuing
a representative mix of concurrent queries against a synthetic catalog while monitoring
average query execution times per query type.  The scaling test dataset size and query
concurrency level are increased each year on a glide path toward the full scale of Data Release 1.

Tasks required for these tests include generation and ingest of each successive test dataset, and
execution of scripts which issue and monitor the suites of representative test queries.

\subsection{Features to be tested}
\label{sec:feat2test}

This version of the \product{} test specification addresses only basic product verification, basic
reliability, and performance/scale testing -- a bare minimum required to conduct ongoing development
and verify that \product{} remains on a realistic path towards meeting its most technically challenging
requirements: those related to successful operability at the scale that will be required by LSST.

\subsection{Features not to be tested}
\label{sec:featnot2test}

Testing of the following are NOT YET COVERED in this specification:

\begin{itemize}
  \item{Fault-tolerance and disaster recovery;}
  \item{Schema evolution;}
  \item{Data ingest performance;}
  \item{Query reproducibility;}
  \item{Cross-match with external datasets.}
\end{itemize}

It is anticipated that test specifications and cases for all of the above will be developed
and added to future revisions of this document.

\subsection{Pass/fail criteria}
\label{sec:passfail}

The results of all tests will be assessed using the criteria described in \citeds{LDM-503} \S4.

\subsection{Suspension criteria and resumption requirements}
\label{suspension}

Refer to individual test cases where applicable.

\subsection{Naming convention}

All tests are named according to the pattern \textsc{prod-scope-xx-yy} where:

\begin{description}[font=\normalfont\scshape]

  \item[prod]{The product code, per \citeds{LDM-294}. Relevant entries for this document are:
    \begin{description}[font=\normalfont\scshape,topsep=-1.0ex]
      \item[qserv]{Qserv distributed database system}
    \end{description}
  }

  \item[scope]{The type of test being described:
    \begin{description}[font=\normalfont\scshape,topsep=-1.0ex]
      \item[acp]{concerning acceptance testing}
      \item[bck]{concerning backup and restore testing}
      \item[fun]{concerning functional testing}
      \item[ins]{concerning installation testing}
      \item[int]{concerning integration testing}
      \item[itf]{concerning interface testing}
      \item[mnt]{concerning maintenance testing}
      \item[prf]{concerning performance testing}
      \item[reg]{concerning regression testing}
      \item[ver]{concerning verification testing}
    \end{description}
  }

  \item[xx]{Test design number (in increments of 10)}
  \item[yy]{Test case number (in increments of 5)}

\end{description}

\section{Test Specification Design}
\subsection{\textsc{QSERV-VER-00}: Qserv Verification}
\label{qserv-ver-00}

\subsubsection{Objective}

This test verifies that \product{} as designed and built meets the overall requirements of the DM system.
Specifically, we verify that:

\begin{itemize}

  \item{Relevant requirements expressed in \citeds{LDM-555} are met by the design;}

  \item{The code as delivered is accompanied by a suite of unit tests;}

  \item{The code as delivered is accompanied by appropriate documentation;}

  \item{The code complies with all relevant DM coding standards\footnote{\url{https://developer.lsst.io/coding/intro.html}};}

  \item{The code makes use of standard DM interfaces to e.g. logging and configuration systems;}

  \item{The code is built and tested by the DM continuous integration system.}

\end{itemize}

\subsubsection{Approach refinements}

The general approach defined in \citeds{LDM-503} is used. Methods include:

\begin{itemize}
  \item{Document inspection;}
  \item{Code inspection;}
  \item{Review of CI system logs.}
\end{itemize}

\subsubsection{Test case identification}

\begin{longtable} {|p{0.4\textwidth}|p{0.6\textwidth}|}\hline
\textbf{Test Case}  & \textbf{Description} \\\hline
\hyperref[qserv-ver-00-00]{\textsc{QSERV-VER-00-00}} & Qserv design inspection \\\hline
\hyperref[qserv-ver-00-05]{\textsc{QSERV-VER-00-05}} & Qserv code inspection \\\hline
\hyperref[qserv-ver-00-10]{\textsc{QSERV-VER-00-10}} & Qserv test inspection \\\hline
\end{longtable}


\newpage
\section{Test Case Specification}

\subsection{Preparation}
\label{sec:prep}

Before running any of the performance test cases, Qserv must be installed on an appropriate test cluster (e.g.
the test machine cluster at CC-IN2P3).  To upgrade Qserv software on the cluster in preparation for testing,
follow directions at \url{http://www.slac.stanford.edu/exp/lsst/qserv/2015_10/HOW-TO/cluster-deployment.html}.

The performance tests will also require an appropriately sized test dataset to be synthesized and ingested,
per the yearly dataset sizing schedule described in \hyperref[qserv-prf-10]{\textsc{QSERV-PRF-10}}.  Tools
for synthesis of ingest of test datasets may be found in the LSST github repot at \url{https://github.com/lsst-dm/db_tests_kpm20}.  Detailed use and context information for the tools is described in \url{https://jira.lsstcorp.org/browse/DM-8405}.

It has also been found that the Qserv shard servers must have engine-independent statistics
loaded for the larger tables in the test dataset, and be properly
configured so that the MariaDB query planner can make use of those statistics.  More information on this
issue is avilable at \url{https://confluence.lsstcorp.org/pages/viewpage.action?pageId=58950786}.

\newpage
\subsection{\textsc{qserv-ver-00-00}: Qserv design inspection}
\label{qserv-ver-00-00}

\subsubsection{Requirements}

\subsubsection{Test items}

\subsubsection{Intercase dependencies}

\subsubsection{Procedure}

\newpage
\subsection{\textsc{QSERV-VER-00-05}: Qserv code inspection}
\label{qserv-ver-00-05}

\subsubsection{Requirements}

\subsubsection{Test items}

This test will check:

\begin{itemize}
  \item{That the code delivered complies with relevant DM software quality standards;}
  \item{That the code is accompanied by appropriate documentation;}
  \item{That the code makes use of appropriate DM interfaces to the rest of the system where applicable;}
  \item{That the code is appropriately tested.}
\end{itemize}

\subsubsection{Intercase dependencies}

None.

\subsubsection{Procedure}

\begin{itemize}

  \item{Check for the existence of a suite of unit test cases accompanying the codebase;}

  \item{Check the code to demonstrate that uses only standardized DM interfaces for such things as logging 
  and configuration (i.e. it does not print directly to screen nor does it contain ad-hoc configuration 
  parsers);}

  \item{Check that the code is accompanied by a user manual describing procedures for its installation and 
  operation.}

\end{itemize}

\newpage
\subsection{\textsc{QSERV-VER-00-10}: Qserv test inspection}
\label{qserv-ver-00-10}

\subsubsection{Requirements}

\subsubsection{Test items}

This test will check:

\begin{itemize}
  \item{That all automated test suites associated with the product pass;}
  \item{That there are no unexpected errors or warnings from the build, test or installation process.}
\end{itemize}

\subsubsection{Intercase dependencies}

\hyperref[qserv-ver-00-05]{QSERV-VER-00-05}.

\subsubsection{Procedure}

Check the logs from the LSST CI system which was used to build and package the software under
test to ensure:

\begin{itemize}

  \item{Successful execution of the test suite, with no failures and no tests being skipped without
  explanatory documentation.}

  \item{That there were no compiler, test, linter or other warnings associated with the software
  build processing.}

\end{itemize}

\newpage
\subsection{\textsc{QSERV-PRF-10-00}: Concurrent Query Performance}
\label{qserv-prf-10-00}

\subsubsection{Requirements \label{sect:reqs}}

DMS-REQ-0356,DMS-REQ-0357.

\subsubsection{Test items}

This test will check that \product{} is able to meet average query completion time targets per query class
under a representative load of simultaneous high and low volume queries while running against an appropriately
scaled test catalog.

\subsubsection{Intercase dependencies}

\hyperref[qserv-prf-20-00]{\textsc{QSERV-PRF-20-00}},
\hyperref[qserv-prf-20-05]{\textsc{QSERV-PRF-20-05}},
\hyperref[qserv-prf-20-10]{\textsc{QSERV-PRF-20-10}},
\hyperref[qserv-prf-20-15]{\textsc{QSERV-PRF-20-15}}.

\subsubsection{Input specification}

\begin{itemize}

  \item{A test catalog of appropriate size (see schedule detail in \hyperref[qserv-prf-10]{\textsc{
  QSERV-PRF-10}}), prepared and ingested into the \product{} instance under test as detailed in
  section~\secref{sec:prep}.}

  \item{The concurrency load execution script, runQueries.py, maintained in the LSST \product{}
  github repository here: \url{https://github.com/lsst/qserv/blob/master/admin/tools/docker/deployment/in2p3/runQueries.py}}

\end{itemize}

\subsubsection{Output specification}

\begin{itemize}
  \item{Log files as generated by the runQueries.py test script.}
\end{itemize}

\subsubsection{Procedure}

\begin{enumerate}

  \item{Inspect and possibly modify the \texttt{CONCURRENCY} and \texttt{TARGET\_RATES} dictionaries in
  the runQueries.py script to adjust the concurrency mix and target execution times per query class.  Query
  mixes and target times are to be adjusted per the following schedule:

    \begin{tabular}{|l|c|c|c|c|c|c|c|}\hline
      \multicolumn{2}{|c|}{\textbf{Query Class}}
        &\textbf{2015}&\textbf{2016}&\textbf{2017}&\textbf{2018}&\textbf{2019}&\textbf{2020}\\\hline
      \multirow{2}{*}{\textbf{LV}}
        &\textbf{\# queries}  & 50 & 60 & 70 & 80 & 90 & 100 \\%\cline{2-8}
        &\textbf{time (sec)}  & 10 & 10 & 10 & 10 & 10 &  10 \\\hline
      \multirow{2}{*}{\textbf{FTSObj}}
        &\textbf{\# queries}  &  3 &  4 &  8 & 12 & 16 &  20 \\%\cline{2-8}
        &\textbf{time (hours)}& 12 &  1 &  1 &  1 &  1 &   1 \\\hline
      \multirow{2}{*}{\textbf{FTSSrc}}
        &\textbf{\# queries}  &  1 &  1 &  2 &  3 &  4 &   5 \\%\cline{2-8}
        &\textbf{time (hours)}& 12 & 12 & 12 & 12 & 12 &  12 \\\hline
      \multirow{2}{*}{\textbf{FTSFSrc}}
        &\textbf{\# queries}  &    &  1 &  2 &  3 &  4 &   5 \\%\cline{2-8}
        &\textbf{time (hours)}&    & 12 & 12 & 12 & 12 &  12 \\\hline
      \multirow{2}{*}{\textbf{joinObjSrc}}
        &\textbf{\# queries}  &  1 &  2 &  4 &  6 &  8 &  10 \\%\cline{2-8}
        &\textbf{time (hours)}& 12 & 12 & 12 & 12 & 12 &  12 \\\hline
      \multirow{2}{*}{\textbf{joinObjFSrc}}
        &\textbf{\# queries}  &    &  1 &  2 &  3 &  4 &   5 \\%\cline{2-8}
        &\textbf{time (hours)}&    & 12 & 12 & 12 & 12 &  12 \\\hline
      \multirow{2}{*}{\textbf{nearN}}
        &\textbf{\# queries}  &    &  1 &  2 &  3 &  4 &   5 \\%\cline{2-8}
        &\textbf{time (hours)}&    &  1 &  1 &  1 &  1 &   1 \\\hline
    \end{tabular}

  }

  \item{Ensure that \product{} instance under test is up to date and that there is no other concurrent
  user activity.}

  \item{Execute the runQueries.py script and let it run for at least 24hrs.}

  \item{Examine log file output and compile performance statistics for the test report.}

\end{enumerate}

\newpage
\subsection{\textsc{QSERV-PRF-20-00}: Object Shared Scan Scaling}
\label{qserv-prf-20-00}

\subsubsection{Requirements}

DMS-REQ-0357.

\subsubsection{Test items}

This test will show that average completion-time of full-scan queries of the Object catalog table grow 
sub-linearly with respect to the number of simultaneously active full-scan queries, within the limits of 
machine resource exhaustion.

\subsubsection{Intercase dependencies}

None.

\subsubsection{Input specification}

\begin{itemize}

  \item{A test catalog of appropriate size (see schedule detail in \hyperref[qserv-prf-10]{\textsc{
  QSERV-PRF-10}}), prepared and ingested into the \product{} instance under test as detailed in 
  section~\secref{sec:prep}.}

  \item{The concurrency load execution script, runQueries.py, maintained in the LSST \product{}
  github repository here: \url{https://github.com/lsst/qserv/blob/master/admin/tools/docker/deployment/in2p3/runQueries.py}}.

\end{itemize}

\subsubsection{Output specification}

\begin{itemize}
  \item{Log files as generated by the runQueries.py test script.}
\end{itemize}

\subsubsection{Procedure}

\hyperref[qserv-prf-scan-scale-test]{\textsc{QSERV-PRF-SCAN-SCALE-TEST}}. 
Query pool of interest is FTSObj.

\newpage
\subsection{\textsc{QSERV-PRF-20-05}: Source Shared Scan Scaling}
\label{qserv-prf-20-05}

\subsubsection{Requirements}

DMS-REQ-0357.

\subsubsection{Test items}

This test will show that average completion-time of full-scan queries of the Source catalog table grow
sub-linearly with respect to the number of simultaneously active full-scan queries, within the limits of
machine resource exhaustion.

\subsubsection{Intercase dependencies}

None.

\subsubsection{Input specification}

\begin{itemize}

  \item{A test catalog of appropriate size (see schedule detail in \hyperref[qserv-prf-10]{\textsc{
  QSERV-PRF-10}}), prepared and ingested into the \product{} instance under test as detailed in
  section~\secref{sec:prep}.}

  \item{The concurrency load execution script, runQueries.py, maintained in the LSST \product{}
  github repository here: \url{https://github.com/lsst/qserv/blob/master/admin/tools/docker/deployment/in2p3/runQueries.py}}

\end{itemize}

\subsubsection{Output specification}

\begin{itemize}
  \item{Log files as generated by the runQueries.py test script.}
\end{itemize}

\subsubsection{Procedure}

\hyperref[qserv-prf-scan-scale-test]{\textsc{QSERV-PRF-SCAN-SCALE-TEST}}.
Query pool of interest is FTSSource.

\newpage
\subsection{\textsc{QSERV-PRF-20-10}: Object Source Join Shared Scan Scaling}
\label{qserv-prf-20-10}

\subsubsection{Requirements}

DMS-REQ-0357.

\subsubsection{Test items}

This test will show that average completion-time of full-scan queries which join the Object and Source 
catalog tables grow sub-linearly with respect to the number of simultaneously active full-scan queries, within 
the limits of machine resource exhaustion.

\subsubsection{Intercase dependencies}

None.

\subsubsection{Input specification}

\begin{itemize}

  \item{A test catalog of appropriate size (see schedule detail in \hyperref[qserv-prf-10]{\textsc{
  QSERV-PRF-10}}), prepared and ingested into the \product{} instance under test as detailed in 
  section~\secref{sec:prep}.}

  \item{The concurrency load execution script, runQueries.py, maintained in the LSST \product{}
  github repository here: \url{https://github.com/lsst/qserv/blob/master/admin/tools/docker/deployment/in2p3/runQueries.py}}

\end{itemize}

\subsubsection{Output specification}

\begin{itemize}
  \item{Log files as generated by the runQueries.py test script.}
\end{itemize}

\subsubsection{Procedure}

\hyperref[qserv-prf-scan-scale-test]{\textsc{QSERV-PRF-SCAN-SCALE-TEST}}.
Query pool of interest is joinObjSrc.

\newpage
\subsection{\textsc{QSERV-PRF-20-15}: Object ForcedSource Join Shared Scan Scaling}
\label{qserv-prf-20-15}

\subsubsection{Requirements}

DMS-REQ-0357.

\subsubsection{Test items}

This test will show that average completion-time of full-scan queries which join the Object and ForcedSource 
catalog tables grow sub-linearly with respect to the number of simultaneously active full-scan queries, within 
the limits of machine resource exhaustion.

\subsubsection{Intercase dependencies}

None.

\subsubsection{Input specification}

\begin{itemize}

  \item{A test catalog of appropriate size (see schedule detail in \hyperref[qserv-prf-10]{\textsc{
  QSERV-PRF-10}}), prepared and ingested into the \product{} instance under test as detailed in 
  section~\secref{sec:prep}.}

  \item{The concurrency load execution script, runQueries.py, maintained in the LSST \product{}
  github repository here: \url{https://github.com/lsst/qserv/blob/master/admin/tools/docker/deployment/in2p3/runQueries.py}}

\end{itemize}

\subsubsection{Output specification}

\begin{itemize}
  \item{Log files as generated by the runQueries.py test script.}
\end{itemize}

\subsubsection{Procedure}

\hyperref[qserv-prf-scan-scale-test]{\textsc{QSERV-PRF-SCAN-SCALE-TEST}}.
Query pool of interest is joinObjFSrc.

\newpage
\section{Test Procedure Specification}
\subsection{QSERV-PRF-SCAN-SCALE-TEST}
\label{qserv-prf-scan-scale-test}

The objective of this procedure is to establish the growth trend for average query execution time of
full-table-scan queries in the pool of interest, as a function of query concurrency.  The test shall
be considered passed if the growth rate is sub-linear (ideally, nearly flat) within the limits of
machine resource exhaustion.  The test procedure is as follows:

\begin{enumerate}

  \item{Ensure that Qserv instance under test is up to date and that there is no other concurrent
  user activity.}

  \item{Inspect and modify the \texttt{CONCURRENCY} and \texttt{TARGET\_RATES} dictionaries in the
  run-Queries.py script. Set \texttt{CONCURRENCY} initially to 1 for the query pool of interest, and to 0
  for all other query pools.  Set \texttt{TARGET\_RATES} for the query pool of interest to the yearly value
  per table in \hyperref[qserv-prf-10-00]{\textsc{QSERV-PRF-10-00}}}.

  \item{\label{execstep}Execute the runQueries.py script and let it run for at least one, but preferably
  several, query cycles.}

  \item{Examine log file output and compile performance statistics to obtain a growth curve point
  for the pool of interest for the test report.}

  \item{Adjust the \texttt{CONCURRENCY} value for the pool of interest and repeat from step \ref{execstep}
  to establish the growth trend and machine resource exhaustion cutoff for the query pool of interest to an
  acceptable degree of accuracy.}

\end{enumerate}




\end{document}
