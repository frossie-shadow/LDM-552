\subsection{QSERV-PRF-SCAN-SCALE-TEST}
\label{qserv-prf-scan-scale-test}

The objective of this procedure is to establish the growth trend for average query execution time of
full-table-scan queries in the pool of interest, as a function of query concurrency.  The test shall
be considered passed if the growth rate is sub-linear (ideally, nearly flat) within the limits of
machine resource exhaustion.  The test procedure is as follows:

\begin{enumerate}

  \item{Ensure that Qserv instance under test is up to date and that there is no other concurrent
  user activity.}

  \item{Inspect and modify the \texttt{CONCURRENCY} and \texttt{TARGET\_RATES} dictionaries in the 
  run-Queries.py script. Set \texttt{CONCURRENCY} initially to 1 for the query pool of interest, and to 0
  for all other query pools.  Set \texttt{TARGET\_RATES} for the query pool of interest to the yearly value
  per table in \hyperref[qserv-prf-10-00]{\textsc{QSERV-PRF-10-00}}}.

  \item{\label{execstep}Execute the runQueries.py script and let it run for at least one, but preferably
  several, query cycles.}

  \item{Examine log file output and compile performance statistics to obtain a growth curve point
  for the pool of interest for the test report.}

  \item{Adjust the \texttt{CONCURRENCY} value for the pool of interest and repeat from step \ref{execstep}
  to establish the growth trend and machine resource exhaustion cutoff for the query pool of interest to an 
  acceptable degree of accuracy.} 

\end{enumerate}
