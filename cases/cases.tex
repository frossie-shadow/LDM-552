\newpage
\section{Test Case Specification}

\subsection{Preparation}
\label{sec:prep}

Before running any of the performance test cases, Qserv must be installed on an appropriate test cluster (e.g. 
the test machine cluster at CC-IN2P3).  To upgrade Qserv software on the cluster in preparation for testing,
follow directions at \url{http://www.slac.stanford.edu/exp/lsst/qserv/2015_10/HOW-TO/cluster-deployment.html}.

The performance tests will also require an appropriately sized test dataset to be synthesized and ingested,
per the yearly dataset sizing schedule described in \hyperref[qserv-prf-10]{\textsc{QSERV-PRF-10}}.  Tools
for synthesis of ingest of test datasets may be found in the LSST github repot at \url{https://github.com/lsst-dm/db_tests_kpm20}.  Detailed use and context information for the tools is described in \url{https://jira.lsstcorp.org/browse/DM-8405}.

It has also been found that the Qserv shard servers must have engine-independent statistics
loaded for the larger tables in the test dataset, and be properly
configured so that the MariaDB query planner can make use of those statistics.  More information on this
issue is avilable at \url{https://confluence.lsstcorp.org/pages/viewpage.action?pageId=58950786}.

\newpage
\subsection{\textsc{qserv-ver-00-00}: Qserv design inspection}
\label{qserv-ver-00-00}

\subsubsection{Requirements}

\subsubsection{Test items}

\subsubsection{Intercase dependencies}

\subsubsection{Procedure}

\newpage
\subsection{\textsc{QSERV-VER-00-05}: Qserv code inspection}
\label{qserv-ver-00-05}

\subsubsection{Requirements}

\subsubsection{Test items}

This test will check:

\begin{itemize}
  \item{That the code delivered complies with relevant DM software quality standards;}
  \item{That the code is accompanied by appropriate documentation;}
  \item{That the code makes use of appropriate DM interfaces to the rest of the system where applicable;}
  \item{That the code is appropriately tested.}
\end{itemize}

\subsubsection{Intercase dependencies}

None.

\subsubsection{Procedure}

\begin{itemize}

  \item{Check for the existence of a suite of unit test cases accompanying the codebase;}

  \item{Check the code to demonstrate that uses only standardized DM interfaces for such things as logging 
  and configuration (i.e. it does not print directly to screen nor does it contain ad-hoc configuration 
  parsers);}

  \item{Check that the code is accompanied by a user manual describing procedures for its installation and 
  operation.}

\end{itemize}

\newpage
\subsection{\textsc{QSERV-VER-00-10}: Qserv test inspection}
\label{qserv-ver-00-10}

\subsubsection{Requirements}

\subsubsection{Test items}

This test will check:

\begin{itemize}
  \item{That all automated test suites associated with the product pass;}
  \item{That there are no unexpected errors or warnings from the build, test or installation process.}
\end{itemize}

\subsubsection{Intercase dependencies}

\hyperref[qserv-ver-00-05]{QSERV-VER-00-05}.

\subsubsection{Procedure}

Check the logs from the LSST CI system which was used to build and package the software under
test to ensure:

\begin{itemize}

  \item{Successful execution of the test suite, with no failures and no tests being skipped without
  explanatory documentation.}

  \item{That there were no compiler, test, linter or other warnings associated with the software
  build processing.}

\end{itemize}

\newpage
\subsection{\textsc{QSERV-PRF-10-00}: Concurrent Query Performance}
\label{qserv-prf-10-00}

\subsubsection{Requirements \label{sect:reqs}}

DMS-REQ-0356,DMS-REQ-0357.

\subsubsection{Test items}

This test will check that \product{} is able to meet average query completion time targets per query class
under a representative load of simultaneous high and low volume queries while running against an appropriately
scaled test catalog.

\subsubsection{Intercase dependencies}

\hyperref[qserv-prf-20-00]{\textsc{QSERV-PRF-20-00}},
\hyperref[qserv-prf-20-05]{\textsc{QSERV-PRF-20-05}},
\hyperref[qserv-prf-20-10]{\textsc{QSERV-PRF-20-10}},
\hyperref[qserv-prf-20-15]{\textsc{QSERV-PRF-20-15}}.

\subsubsection{Input specification}

\begin{itemize}

  \item{A test catalog of appropriate size (see schedule detail in \hyperref[qserv-prf-10]{\textsc{
  QSERV-PRF-10}}), prepared and ingested into the \product{} instance under test as detailed in
  section~\secref{sec:prep}.}

  \item{The concurrency load execution script, runQueries.py, maintained in the LSST \product{}
  github repository here: \url{https://github.com/lsst/qserv/blob/master/admin/tools/docker/deployment/in2p3/runQueries.py}}

\end{itemize}

\subsubsection{Output specification}

\begin{itemize}
  \item{Log files as generated by the runQueries.py test script.}
\end{itemize}

\subsubsection{Procedure}

\begin{enumerate}

  \item{Inspect and possibly modify the \texttt{CONCURRENCY} and \texttt{TARGET\_RATES} dictionaries in
  the runQueries.py script to adjust the concurrency mix and target execution times per query class.  Query
  mixes and target times are to be adjusted per the following schedule:

    \begin{tabular}{|l|c|c|c|c|c|c|c|}\hline
      \multicolumn{2}{|c|}{\textbf{Query Class}}
        &\textbf{2015}&\textbf{2016}&\textbf{2017}&\textbf{2018}&\textbf{2019}&\textbf{2020}\\\hline
      \multirow{2}{*}{\textbf{LV}}
        &\textbf{\# queries}  & 50 & 60 & 70 & 80 & 90 & 100 \\%\cline{2-8}
        &\textbf{time (sec)}  & 10 & 10 & 10 & 10 & 10 &  10 \\\hline
      \multirow{2}{*}{\textbf{FTSObj}}
        &\textbf{\# queries}  &  3 &  4 &  8 & 12 & 16 &  20 \\%\cline{2-8}
        &\textbf{time (hours)}& 12 &  1 &  1 &  1 &  1 &   1 \\\hline
      \multirow{2}{*}{\textbf{FTSSrc}}
        &\textbf{\# queries}  &  1 &  1 &  2 &  3 &  4 &   5 \\%\cline{2-8}
        &\textbf{time (hours)}& 12 & 12 & 12 & 12 & 12 &  12 \\\hline
      \multirow{2}{*}{\textbf{FTSFSrc}}
        &\textbf{\# queries}  &    &  1 &  2 &  3 &  4 &   5 \\%\cline{2-8}
        &\textbf{time (hours)}&    & 12 & 12 & 12 & 12 &  12 \\\hline
      \multirow{2}{*}{\textbf{joinObjSrc}}
        &\textbf{\# queries}  &  1 &  2 &  4 &  6 &  8 &  10 \\%\cline{2-8}
        &\textbf{time (hours)}& 12 & 12 & 12 & 12 & 12 &  12 \\\hline
      \multirow{2}{*}{\textbf{joinObjFSrc}}
        &\textbf{\# queries}  &    &  1 &  2 &  3 &  4 &   5 \\%\cline{2-8}
        &\textbf{time (hours)}&    & 12 & 12 & 12 & 12 &  12 \\\hline
      \multirow{2}{*}{\textbf{nearN}}
        &\textbf{\# queries}  &    &  1 &  2 &  3 &  4 &   5 \\%\cline{2-8}
        &\textbf{time (hours)}&    &  1 &  1 &  1 &  1 &   1 \\\hline
    \end{tabular}

  }

  \item{Ensure that \product{} instance under test is up to date and that there is no other concurrent
  user activity.}

  \item{Execute the runQueries.py script and let it run for at least 24hrs.}

  \item{Examine log file output and compile performance statistics for the test report.}

\end{enumerate}

\newpage
\subsection{\textsc{QSERV-PRF-20-00}: Object Shared Scan Scaling}
\label{qserv-prf-20-00}

\subsubsection{Requirements}

DMS-REQ-0357.

\subsubsection{Test items}

This test will show that average completion-time of full-scan queries of the Object catalog table grow 
sub-linearly with respect to the number of simultaneously active full-scan queries, within the limits of 
machine resource exhaustion.

\subsubsection{Intercase dependencies}

None.

\subsubsection{Input specification}

\begin{itemize}

  \item{A test catalog of appropriate size (see schedule detail in \hyperref[qserv-prf-10]{\textsc{
  QSERV-PRF-10}}), prepared and ingested into the \product{} instance under test as detailed in 
  section~\secref{sec:prep}.}

  \item{The concurrency load execution script, runQueries.py, maintained in the LSST \product{}
  github repository here: \url{https://github.com/lsst/qserv/blob/master/admin/tools/docker/deployment/in2p3/runQueries.py}}.

\end{itemize}

\subsubsection{Output specification}

\begin{itemize}
  \item{Log files as generated by the runQueries.py test script.}
\end{itemize}

\subsubsection{Procedure}

\hyperref[qserv-prf-scan-scale-test]{\textsc{QSERV-PRF-SCAN-SCALE-TEST}}. 
Query pool of interest is FTSObj.

\newpage
\subsection{\textsc{QSERV-PRF-20-05}: Source Shared Scan Scaling}
\label{qserv-prf-20-05}

\subsubsection{Requirements}

DMS-REQ-0357.

\subsubsection{Test items}

This test will show that average completion-time of full-scan queries of the Source catalog table grow
sub-linearly with respect to the number of simultaneously active full-scan queries, within the limits of
machine resource exhaustion.

\subsubsection{Intercase dependencies}

None.

\subsubsection{Input specification}

\begin{itemize}

  \item{A test catalog of appropriate size (see schedule detail in \hyperref[qserv-prf-10]{\textsc{
  QSERV-PRF-10}}), prepared and ingested into the \product{} instance under test as detailed in
  section~\secref{sec:prep}.}

  \item{The concurrency load execution script, runQueries.py, maintained in the LSST \product{}
  github repository here: \url{https://github.com/lsst/qserv/blob/master/admin/tools/docker/deployment/in2p3/runQueries.py}}

\end{itemize}

\subsubsection{Output specification}

\begin{itemize}
  \item{Log files as generated by the runQueries.py test script.}
\end{itemize}

\subsubsection{Procedure}

\hyperref[qserv-prf-scan-scale-test]{\textsc{QSERV-PRF-SCAN-SCALE-TEST}}.
Query pool of interest is FTSSource.

\newpage
\subsection{\textsc{QSERV-PRF-20-10}: Object Source Join Shared Scan Scaling}
\label{qserv-prf-20-10}

\subsubsection{Requirements}

DMS-REQ-0357.

\subsubsection{Test items}

This test will show that average completion-time of full-scan queries which join the Object and Source 
catalog tables grow sub-linearly with respect to the number of simultaneously active full-scan queries, within 
the limits of machine resource exhaustion.

\subsubsection{Intercase dependencies}

None.

\subsubsection{Input specification}

\begin{itemize}

  \item{A test catalog of appropriate size (see schedule detail in \hyperref[qserv-prf-10]{\textsc{
  QSERV-PRF-10}}), prepared and ingested into the \product{} instance under test as detailed in 
  section~\secref{sec:prep}.}

  \item{The concurrency load execution script, runQueries.py, maintained in the LSST \product{}
  github repository here: \url{https://github.com/lsst/qserv/blob/master/admin/tools/docker/deployment/in2p3/runQueries.py}}

\end{itemize}

\subsubsection{Output specification}

\begin{itemize}
  \item{Log files as generated by the runQueries.py test script.}
\end{itemize}

\subsubsection{Procedure}

\hyperref[qserv-prf-scan-scale-test]{\textsc{QSERV-PRF-SCAN-SCALE-TEST}}.
Query pool of interest is joinObjSrc.

\newpage
\subsection{\textsc{QSERV-PRF-20-15}: Object ForcedSource Join Shared Scan Scaling}
\label{qserv-prf-20-15}

\subsubsection{Requirements}

DMS-REQ-0357.

\subsubsection{Test items}

This test will show that average completion-time of full-scan queries which join the Object and ForcedSource 
catalog tables grow sub-linearly with respect to the number of simultaneously active full-scan queries, within 
the limits of machine resource exhaustion.

\subsubsection{Intercase dependencies}

None.

\subsubsection{Input specification}

\begin{itemize}

  \item{A test catalog of appropriate size (see schedule detail in \hyperref[qserv-prf-10]{\textsc{
  QSERV-PRF-10}}), prepared and ingested into the \product{} instance under test as detailed in 
  section~\secref{sec:prep}.}

  \item{The concurrency load execution script, runQueries.py, maintained in the LSST \product{}
  github repository here: \url{https://github.com/lsst/qserv/blob/master/admin/tools/docker/deployment/in2p3/runQueries.py}}

\end{itemize}

\subsubsection{Output specification}

\begin{itemize}
  \item{Log files as generated by the runQueries.py test script.}
\end{itemize}

\subsubsection{Procedure}

\hyperref[qserv-prf-scan-scale-test]{\textsc{QSERV-PRF-SCAN-SCALE-TEST}}.
Query pool of interest is joinObjFSrc.
