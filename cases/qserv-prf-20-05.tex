\subsection{\textsc{QSERV-PRF-20-05}: Source Shared Scan Scaling}
\label{qserv-prf-20-05}

\subsubsection{Requirements}

DMS-REQ-0357.

\subsubsection{Test items}

This test will show that average completion-time of full-scan queries of the Source catalog table grow
sub-linearly with respect to the number of simultaneously active full-scan queries, within the limits of
machine resource exhaustion.

\subsubsection{Intercase dependencies}

None.

\subsubsection{Input specification}

\begin{itemize}

  \item{A test catalog of appropriate size (see schedule detail in \hyperref[qserv-prf-10]{\textsc{
  QSERV-PRF-10}}), prepared and ingested into the \product{} instance under test as detailed in
  section~\secref{sec:prep}.}

  \item{The concurrency load execution script, runQueries.py, maintained in the LSST \product{}
  github repository here: \url{https://github.com/lsst/qserv/blob/master/admin/tools/docker/deployment/in2p3/runQueries.py}}

\end{itemize}

\subsubsection{Output specification}

\begin{itemize}
  \item{Log files as generated by the runQueries.py test script.}
\end{itemize}

\subsubsection{Procedure}

\hyperref[qserv-prf-scan-scale-test]{\textsc{QSERV-PRF-SCAN-SCALE-TEST}}.
Query pool of interest is FTSSource.
