\subsection{\textsc{QSERV-PRF-10}: Qserv Concurrent Query Performance}
\label{qserv-prf-10}

\subsubsection{Objective}

This test design verifies that Qserv meets query concurrency performance requirements
per \citeds{LSE-61} and \citeds{LDM-555}.

\subsubsection{Approach refinements}

Query load on Qserv is anticipated to be a combination of "low volume" queries (queries that touch a
small area of sky, or request a small number of objects) and "high volume" queries (queries that involve
full-sky scans and may involve more complex spatial and temporal correlations).

Concurrency performance targets are expressed in terms of the allowable numbers of each of these query
types that may be simultaneously active within the system while still meeting specified average query
completion times per type.

The number of simultaneous low and high volume queries and the size of the dataset against which the
queries are issued are evolved along a glide path toward the eventual operation targets, provide a
sequence of "data challenge" tests.  The following schedule shall be followed (FY20 LV and HV targets
taken from \citeds{LSE-61}):

\begin{longtable}{|l|l|l|l|}\hline
  \textbf{Year}&\textbf{Dataset Size}&\textbf{\# LV Queries}&\textbf{\# HV Queries}\endhead\hline
  2015 & 10\% DR1  & 50  & 5  \\\hline
  2016 & 20\% DR1  & 60  & 10 \\\hline
  2017 & 30\% DR1  & 70  & 20 \\\hline
  2018 & 50\% DR1  & 80  & 30 \\\hline
  2019 & 75\% DR1  & 90  & 40 \\\hline
  2020 & 100\% DR1 & 100 & 50 \\\hline
\end{longtable}

\subsubsection{Test case identification}

\begin{longtable} {|p{0.4\textwidth}|p{0.6\textwidth}|}\hline
\textbf{Test Case}  & \textbf{Description} \\\hline
\hyperref[qserv-prf-10-00]{\textsc{QSERV-PRF-10-00}} & Qserv Concurrent Query Performance \\\hline
\end{longtable}
