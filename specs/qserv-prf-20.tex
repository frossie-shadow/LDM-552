\subsection{\textsc{QSERV-PRF-20}: Qserv Shared Scan Scaling}
\label{qserv-prf-20}

\subsubsection{Objective}

This test design verifies that the shared-scan feature of Qserv functions per design in \citeds{LDM-135},
allowing average completion-time of full-scan queries to grow sub-linearly with respect to the number of
simultaneously active full-scan queries, within the limits of machine resource exhaustion.

Proper behavior of Qserv in this regard is an important sub-goal of being able to feasibly meet high 
volume concurrency requirements as specified in \citeds{LSE-61}.  The outcome of these tests will also
help refine machine resource provisioning targets for operations.

\subsubsection{Approach refinements}

Full-scan queries will be simultaneously issued at increasing levels of concurrency while query 
completion times are monitored to establish the average query completion time growth trend.  The
tests will be carried out with full-scan queries of various types and levels of complexity in order
to verify that shared scan query classification and per-type scan scheduling function per design. 

\subsubsection{Test case identification}

\begin{longtable} {|p{0.4\textwidth}|p{0.6\textwidth}|}\hline
\textbf{Test Case}  & \textbf{Description} \\\hline
\hyperref[qserv-prf-20-00]{\textsc{QSERV-PRF-20-00}} & Object Shared Scan Scaling \\\hline
\hyperref[qserv-prf-20-05]{\textsc{QSERV-PRF-20-05}} & ObjectExtra Shared Scan Scaling \\\hline
\hyperref[qserv-prf-20-10]{\textsc{QSERV-PRF-20-10}} & Object Source Join Shared Scan Scaling \\\hline
\hyperref[qserv-prf-20-15]{\textsc{QSERV-PRF-20-15}} & Object FourcedSource Join Shared Scan Scaling \\\hline
\end{longtable}
