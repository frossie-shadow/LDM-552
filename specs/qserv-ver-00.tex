\subsection{\textsc{QSERV-VER-00}: Qserv Verification}
\label{qserv-ver-00}

\subsubsection{Objective}

This test verifies that \product{} as designed and built meets the overall requirements of the DM system.
Specifically, we verify that:

\begin{itemize}

  \item{Relevant requirements expressed in \citeds{LDM-555} are met by the design;}

  \item{The code as delivered is accompanied by a suite of unit tests;}

  \item{The code as delivered is accompanied by appropriate documentation;}

  \item{The code complies with all relevant DM coding standards\footnote{\url{https://developer.lsst.io/coding/intro.html}};}

  \item{The code makes use of standard DM interfaces to e.g. logging and configuration systems;}

  \item{The code is built and tested by the DM continuous integration system.}

\end{itemize}

\subsubsection{Approach refinements}

The general approach defined in \citeds{LDM-503} is used. Methods include:

\begin{itemize}
  \item{Document inspection;}
  \item{Code inspection;}
  \item{Review of CI system logs.}
\end{itemize}

\subsubsection{Test case identification}

\begin{longtable} {|p{0.4\textwidth}|p{0.6\textwidth}|}\hline
\textbf{Test Case}  & \textbf{Description} \\\hline
\hyperref[qserv-ver-00-00]{\textsc{QSERV-VER-00-00}} & Qserv design inspection \\\hline
\hyperref[qserv-ver-00-05]{\textsc{QSERV-VER-00-05}} & Qserv code inspection \\\hline
\hyperref[qserv-ver-00-10]{\textsc{QSERV-VER-00-10}} & Qserv test inspection \\\hline
\end{longtable}
